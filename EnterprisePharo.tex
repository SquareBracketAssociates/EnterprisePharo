%!TEX TS-program = lualatex
% -*- mode: LaTeX; -*-
\documentclass[10pt,twoside,english,showtrims]{support/latex/sbabook/sbabook}
\let\wholebook=\relax

\usepackage{import}
\subimport{support/latex/}{common.tex}

%=================================================================
% Add the path for the figures of each chapter here:
\graphicspath{
        {../figures/}
        {../Copyright/}
        {../PillarChap/}
        {../Artefact/}
        {../Zinc-HTTP-Client/}
        {../Zinc-HTTP-Server/}
        {../Zinc-Encoding-Meta/}
        {../Zinc-REST/}
        {../Fuel/}
        {../STON/}
        {../Voyage/}
        {../WebApp/}
        {../WebSockets/}
        {../NeoJSON/}
        {../RenoirST/}
        {../NeoCSV/}
        {../Teapot/}
        {../Mustache/}
}
%=================================================================
\title{Enterprise Pharo\titlebreak{:} a Web Perspective}
\author{
    Damien Cassou \and
    Stéphane Ducasse \and
    Luc Fabresse \and
    Johan Fabry \and
    Sven Van Caekenberghe}
\date{\gitdate\titlebreak[\smallskip]{ -- }\protect\gitCommitInfo}

\hypersetup{pdfinfo = {
    Title = {Enterprise Pharo: a Web Perspective},
    Author = {Damien Cassou, Stéphane Ducasse, Luc Fabresse, Johan Fabry, Sven Van Caekenberghe},
    Keywords = {pharo, smalltalk}}}
% =================================================================
\begin{document}

% Title page and colophon on verso
\maketitle
\pagestyle{titlingpage}
\thispagestyle{titlingpage} % \pagestyle does not work on the first one…

\cleartoverso
{\small

  Copyright 2015 by Damien Cassou, Stéphane Ducasse, Luc Fabresse, Johan Fabry,
  and Sven Van Caekenberghe.

  The contents of this book are protected under the Creative Commons
  Attribution-ShareAlike 3.0 Unported license.

  You are \textbf{free}:
  \begin{itemize}
  \item to \textbf{Share}: to copy, distribute and transmit the work,
  \item to \textbf{Remix}: to adapt the work,
  \end{itemize}

  Under the following conditions:
  \begin{description}
  \item[Attribution.] You must attribute the work in the manner specified by the
    author or licensor (but not in any way that suggests that they endorse you
    or your use of the work).
  \item[Share Alike.] If you alter, transform, or build upon this work, you may
    distribute the resulting work only under the same, similar or a compatible
    license.
  \end{description}

  For any reuse or distribution, you must make clear to others the
  license terms of this work. The best way to do this is with a link to
  this web page: \\
  \url{http://creativecommons.org/licenses/by-sa/3.0/}

  Any of the above conditions can be waived if you get permission from
  the copyright holder. Nothing in this license impairs or restricts the
  author's moral rights.

  \begin{center}
    \includegraphics[width=0.2\textwidth]{CreativeCommons-BY-SA.pdf}
  \end{center}

  Your fair dealing and other rights are in no way affected by the
  above. This is a human-readable summary of the Legal Code (the full
  license): \\
  \url{http://creativecommons.org/licenses/by-sa/3.0/legalcode}

  \vfill

  \begin{tabular}{@{}c@{\quad}l}
    \multirow{2}{*}{\includegraphics[width=2em]{support/latex/sbabook/sba-logo.pdf}}
    & Published by Square Bracket Associates, Switzerland. \\
    & \url{http://squarebracketassociates.org} \\[\smallskipamount]
    & ISBN xx-xxx-xx-xxx-xxxx-xxxx- \\
    & First Edition, December 2015. \\
  \end{tabular}
  \medskip

  Book layout and typography based on the \code{sbabook} \LaTeX{} class by Damien Pollet.
}


\frontmatter
\pagestyle{plain}


\chapter*{About this book}

\emph{Enterprise Pharo} is the third volume of the series, following
\emph{Pharo by Example} and \emph{Deep into Pharo}. It covers
enterprise libraries and frameworks, and in particular those useful for
doing web development.

The book is structured in five parts.\\
\textbf{The first part} talks about simple web applications, starting
with a minimal web application in chapter 1 on Teapot and then a
tutorial on building a more complete web application in chapter 2.\\
\textbf{Part two of the book} deals with HTTP support in Pharo,
talking about character encoding in chapter 3, about using Pharo as an
HTTP Client (in chapter 4) and server (in chapter 5), and about using
WebSockets (in chapter 6).\\
\textbf{In the third part} we discuss the handling of data for the
application. Firstly we treat data that is in the form of
comma-separated values (CSV) in chapter 7. Secondly and thirdly, we
treat JSON (in chapter 8) and its Smalltalk counterpart STON (in
chapter 9). Fourthly, serialization and deserialization of object
graphs with Fuel is treated in chapter 10. Lastly, we discuss the
Voyage persistence framework and persisting to MongoDB databases in
chapter 11.\\
\textbf{Part four of the book} deals with the presentation layer.
Chapter 12 shows how to use Mustache templates in Pharo, and chapter
13 talks about programmatic generation of CSS files. The documentation
of applications could be written in Pillar, which is presented in
chapter 14. How to generate .pdf files from the application with
Artefact is shown in chapter 15.\\
\textbf{The fifth part of the book} deals with deploying the web
application. This is explained in chapter 16 that talks not only about
how to build and run the application, but also other important topics
like monitoring.

\paragraph{This book is a collective work}
The editors have curated and reformatted the following chapters from blog posts
and tutorials written by many people. Here is the complete list of contributors
to the book, in alphabetical order:

\begin{multicols}{3}
Olivier Auverlot\\
Sven Van Caekenberghe\\
Damien Cassou\\
Gabriel Cotelli\\
Christophe Demarey\\
Martín Dias\\
Stéphane Ducasse\\
Luc Fabresse\\
Johan Fabry\\
Cyril Ferlicot Delbecque\\
Norbert Hartl\\
Guillaume Larchevêque\\
Max Leske\\
Esteban Lorenzano\\
Attila Magyar\\
Mariano Martinez-Peck\\
Damien Pollet\\
\end{multicols}

\tableofcontents*
\clearpage\listoffigures

\mainmatter

\part{Simple Web applications}
\input{Teapot/Teapot}
\input{WebApp/WebApp}

\part{HTTP}
\input{Zinc-Encoding-Meta/Zinc-Encoding-Meta}
\input{Zinc-HTTP-Client/Zinc-HTTP-Client}
\input{Zinc-HTTP-Server/Zinc-HTTP-Server}
\input{WebSockets/WebSockets}

\part{Data}
\input{NeoCSV/NeoCSV}
\input{NeoJSON/NeoJSON}
\input{STON/STON}
\input{Fuel/Fuel}
\input{Voyage/Voyage}

\part{Presentation}
\input{Mustache/Mustache}
\input{RenoirST/RenoirST}
\input{PillarChap/Pillar}
\input{Artefact/Artefact}

\part{Deployment}
\input{DeploymentWeb/DeployForProduction}

\backmatter

% \chapter*{Pages to print in color}
% % ch1 Teapot
% Page numbers: \pageref{TeapotWelcome}, %
% \pageref{plainText}, %
% % ch2 WebApp
% \pageref{imageweb}, %
% \pageref{pharo-in-action}, %
% \pageref{helloWebApp-in-action}, %
% \pageref{InspectingIncomingRequest}, %
% \pageref{htmlResponse}, %
% \pageref{servingPharoLogo}, %
% \pageref{ChangingDisplayedImage}, %
% \pageref{dnu}, %
% \pageref{live}, %
% \pageref{CreateTestClass}, %
% \pageref{RunningTestCase}, %
% \pageref{OpenMC}, %
% \pageref{MCOpened}, %
% \pageref{WebAppStHubRepo}, %
% \pageref{AddRepoInMC}, %
% \pageref{StHubRepoInMc}, %
% \pageref{runningOnCloud9}


\end{document}

%=================================================================
%%% Local Variables:
%%% coding: utf-8
%%% mode: latex
%%% TeX-master: t
%%% TeX-PDF-mode: t
%%% ispell-local-dictionary: "english"
%%% End:
